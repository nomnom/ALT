\section{Abstract}

In the following work we present a system for capturing ambient light in a real scene and recreating it in a room equipped with computer-controlled lamps.
We capture incident light in one point of a scene with a simple light probe consisting of a camera and a reflective sphere.
The acquired environment map is then transferred to a room where an approximated lighting condition is recreated with multiple LED lamps. 
To achieve this, we first measure the impact each lamp has on the illumination with a light probe and acquire one image per lamp.
A linear combination of these images produces a new environment map, which we can recreate inside the room by setting the intensities of the lamps.
We employ Quadratic Programming to find the linear combination that approximates a given environment map best.
We speed up the optimization process by downsampling the light probe data, which reduces the dimension of our problem.
Our method is fast enough for real-time light transfer and works with all types of linear controlled illuminants.
In order to evaluate our method, we designed and constructed an omnidirectional lighting system that can spatially illuminate a room in full-color.
We first explore several different configurations of our lighting system, our sampling and our optimization algorithm.
We then demonstrate our method's capabilities by capturing static and dynamic ambient light in one location and transferring it into a room.

\section{Kurzfassung}

In der folgenden Arbeit stellen wir ein System vor mit dem man das Umgebungslicht an einem Ort aufnehmen und in einen Raum mithilfe computergesteuerter Lampen nachstellen kann. 
Wir verwenden eine einfache Light Probe, bestehend aus einer Kamera und einer verspiegelten Kugel um das Licht das in einem Punkt einer Szene eingeht aufzunehmen.
Dies liefert uns eine Environment Map welche dann in einem Raum durch das Ansteuern von LED-Lampen approximiert wird.
Dazu messen wir zuerst die Lichtverteilung jeder einzelnen Lampen im Raum mit einer Light Probe und nehmen ein Bild pro Lampe auf.
Durch eine Linearkombination können wir neue Environment Maps erzeugen, welche wir dann im Raum durch Einstellen der Lampenhelligkeit darstellen können.
Wir verwenden Quadratic Programming um eine Linearkombination zu finden, welche die zu übertragende Environment Map am besten approximiert.
Wir beschleunigen den Optimierungsprozess indem wir die Environment Maps abtasten und somit die Größe des Problems reduzieren.
Unsere Methode eignet sich für das Übertragen von Umgebungslicht in Echtzeit und funktioniert mit jeder linear steuerbaren Lichtquelle.
Für die Evaluierung haben wir neben einer mobilen Light Probe auch ein Beleuchtungssystem entworfen und aufgebaut. 
Es handelt sich um eine mobile Konstruktion mit der ein Raum omnidirektional und farbig ausgeleuchtet werden kann.
Wir untersuchen zuerst unterschiedliche Konfigurationen unseres Beleuchtungssystems, des Abtastprozesses und der Optimierung.
Anschließend demonstrieren wird die Fähigkeiten unseres Systems in dem wir statische und dynamische Umgebungsbeleuchtungen aufnehmen und in einem Raum wiedergeben.

 
