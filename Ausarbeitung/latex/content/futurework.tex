\chapter{Future Work}
  We have shown that our lighting system can illuminate a room omnidirectional and reproduce the global impression of environment maps.
  We presumed our LEDs are linear in the input data and mentioned in section \ref{sec:ledlimitations} that this is not the case. 
  The color channels showed an interdependency and the output of the red channel depends on the output of the blue and green channel.
  An additional calibration could alleviate this effect. 
  We could for example measure the light color of the LEDs for all possible combinations of input data and derive a model that linearizes the response.
  
  We have shown our method works with indirect lighting and we can imagine an extension to direct illumination. 
  This not only requires a lighting system capable of direct illumination, for example a room-sized light stage \cite{debevec2002lighting}, 
  but also further investigation regarding the light probe.
  If we want to capture direct light in addition to the indirect light, we must use a camera with a much higher dynamic range.
  For the Calibration Probe, we can take multiple images  with different exposures and reconstruct an image with a high dynamic range (HDR) \cite{debevec2008recovering}.
  For the Capture Probe, which has to record in real-time, we could use an expensive HDR video camera. 
  A simpler solution was proposed by Waese \cite{waese2009real} and uses a special prism with shaded surfaces to record multiple exposures of the light probe object with a single image.
  
