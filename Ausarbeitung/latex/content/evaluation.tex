\chapter{Evaluation}
\label{chap:evaluation}
  For evaluation we set up our lighting system in a medium sized room.
  The room was not optimal and had rather bad reflective properties: The whole ceiling was lined with metal ducts and reflectors of the original room lighting which caused strong specular reflections.
  Several dark-blue shelves absorbed so much light that we had to drape them in white table-cloth.
  Figure \ref{fig:plabsetup} shows the placement of the lighting system in the room.
     
  All experiments were conducted on a Linux machine located in the same room. 
  It was equipped with a Core i7 920 CPU  running at 2.67 GHz and had 12GB of memory.
  
  
  \begin{figure}[H]
    \subfloat[Grace\label{fig:env_grace}]{
      \includegraphics[width=0.15\linewidth]{../../resources/main/grace.png}
    }
    \hfill
    \subfloat[Outdoor\label{fig:env_canon}]{
      \includegraphics[width=0.15\linewidth]{../../resources/main/canon3.png}
    }
    \hfill
    \subfloat[Dots\label{fig:env_dots}]{
      \includegraphics[width=0.15\linewidth]{../../resources/main/dots.png}
    }
    \hfill
    \subfloat[CMY\label{fig:env_cmy}]{
      \includegraphics[width=0.15\linewidth]{../../resources/main/cmy.png}
    }
    \caption[Static targets]{The static environment maps we used for the transfer: The Grace HDR light probe from Debevec (a), an outdoor scene (b) captured with our Canon light probe and two hand-drawn extreme cases (c), (d). }{}
    
    \label{fig:envmaps}
  \end{figure}
  
  We tested our system with the four static environment maps shown in figure \ref{fig:envmaps}. 
  The first one (a) is a High Dynamic Range light probe taken inside the \emph{Grace} cathedral by Debevec \cite{PROBES}. 
  The second one (b) was captured with our Canon light probe and shows a sun-lit \emph{Outdoor} scene without direct illumination.
  We also include two hand-drawn targets \emph{Dots} (c) and \emph{CMY} (d) as extreme cases to demonstrate the limits of our omnidirectional lighting system.
  Additionally, we recorded a dynamic environment map (88 frames) with the Webcam Probe by moving it through a colorful scene.
  Five frames are shown in table \ref{tab:main_dynamic}.
    
  To measure the quality of our results we took light probe images of the produced illumination with our Canon Probe and compared them to the targets.
  We calculate the \emph{result error} $e_r$ by taking the sum of squared differences over all color channels of the available pixels $P$ and divide it by $3|P|$.
  Masked pixels and pixels outside our sampling range are ignored.
  The \emph{difference error} $e_d$ is calculated the same way, but is used to compare two results.
  \newpage
  We will show the values of our lamps in form of a small image as follows:
   \begin{figure}[H]
    \label{fig:lampimage}
    \includegraphics[width=0.4\linewidth]{../../resources/lampimage.png}
   \end{figure}
    
  We split the evaluation in four parts:
  First, we measure how our the choice of the \emph{segment sizes} (and thus the \emph{number of lamps} $|L|$) influences the result error $e_r$.
  Second, we show how our nearest-neighbor approach performs for different \emph{number of sampling directions} $|D|$ and compare it to the limit case where the neighborhood consist of a single pixel.
  Third, we take a look at the run-time of the optimizer and show how it is linked to the \emph{number of lamps} and \emph{number of sampling directions}. 
  Fourth, we show how the \emph{target scaling} $s$ affects the quality of the results.
  Based on this analysis, we select a lamp and sampling configuration that produces good results with real-time speeds.

\section{Segment Size}
 \label{eval:segsize}
  We evaluated segment sizes of 10/20/40/60/120 LEDs, which gave us 96/48/24/16/8 RGB Lamps respectively, to see if smaller segments produce better results.
  We also included  \emph{uniform lighting} (1 lamp) where all LEDs display the same color.
  Note that in the case of 8 lamps the walls and ceiling cannot be illuminated independently because each stick acts as a single lamp.
  
  Our assumption was that increasing the number of lamps increases the quality of the result, but only up to some point because our lighting system has a limited spatial resolution.
    
    \npdecimalsign{.}
    \nprounddigits{3}
    \begin{table}[H]
     \caption[Evaluation of lamp configurations (result error)]{\label{tab:segsize} Result error $e_r$ for several lamp configurations. No downsampling was performed.}
     \begin{tabular}{c|n{2}{3}|n{2}{3}|n{2}{3}|n{2}{3}|n{2}{3}}
        \toprule 
        {|L|} & {Outdoor} & {Grace} & {Dots} & {CMY} & {Video (avg.)} \\
        \midrule
        \input ../../Results/tables/segsize.table
        \bottomrule
     \end{tabular}
    \end{table}
    \npnoround
    
  Table \ref{tab:segsize} shows the result error $e_r$ for our six different lamp configurations and all five target environment maps.
  For the video environment map, we selected 9 frames (spaced 10 frames apart) and averaged the error.
  We did not perform any downsampling and used all available pixels in order to be independent from sampling artifacts.

  We recorded the calibration data only once for the smallest segment size. 
  All other configurations are derived by adding up the radiance to form larger segments.
  This avoids variations in the calibration data and keeps the results comparable.
  
  The table clearly shows that a spatial illumination with multiple lamps always performs better than uniform lighting.
  It also reveals that a setup with 48 or 96 lamps produces only a marginal improvement over 24 lamps.   
   
     
    \begin{table}[H]
     \caption[Evaluation of lamp configurations (images) ]{\label{tab:grace_segsize} Results and lamp values for the Grace target created with different lamp configurations.
      Second row: absolute difference between two successive results. Third row: the difference error $e_d$ between two successive results.  }
     \begin{tabular}{cccccc}
        \toprule 
        $|L|=1$ & $|L|=8$ & $|L|=16$ & $|L|=24$ & $|L|=48$ & $|L|=96$ \\
        \midrule
        
        \includegraphics[width=0.12\linewidth]{../../resources/segsize/grace_segsize_uni.png}
        & \includegraphics[width=0.12\linewidth]{../../resources/segsize/grace_segsize_x120i.png}
        & \includegraphics[width=0.12\linewidth]{../../resources/segsize/grace_segsize_x60i.png}
        & \includegraphics[width=0.12\linewidth]{../../resources/segsize/grace_segsize_x40i.png}
        & \includegraphics[width=0.12\linewidth]{../../resources/segsize/grace_segsize_x20i.png}
        & \includegraphics[width=0.12\linewidth]{../../resources/segsize/grace_segsize_x10.png} \\
        
          \includegraphics[height=0.06\linewidth,width=0.12\linewidth]{../../resources/segsize/grace_segsize_uni_lamps.png}
        & \includegraphics[height=0.06\linewidth,width=0.12\linewidth]{../../resources/segsize/grace_segsize_x120i_lamps.png}
        & \includegraphics[height=0.06\linewidth,width=0.12\linewidth]{../../resources/segsize/grace_segsize_x60i_lamps.png}
        & \includegraphics[height=0.06\linewidth,width=0.12\linewidth]{../../resources/segsize/grace_segsize_x40i_lamps.png}
        & \includegraphics[height=0.06\linewidth,width=0.12\linewidth]{../../resources/segsize/grace_segsize_x20i_lamps.png}
        & \includegraphics[height=0.06\linewidth,width=0.12\linewidth]{../../resources/segsize/grace_segsize_x10_lamps.png} \\
        
     \end{tabular}
     
      \begin{tabular}{ccccc}
        
         \includegraphics[width=0.12\linewidth]{../../resources/segsize/grace_segsize_x120i_interdiff.png}
        & \includegraphics[width=0.12\linewidth]{../../resources/segsize/grace_segsize_x60i_interdiff.png}
        & \includegraphics[width=0.12\linewidth]{../../resources/segsize/grace_segsize_x40i_interdiff.png}
        & \includegraphics[width=0.12\linewidth]{../../resources/segsize/grace_segsize_x20i_interdiff.png}
        & \includegraphics[width=0.12\linewidth]{../../resources/segsize/grace_segsize_x10_interdiff.png}  \\
        \hline
        12.82e-05 & 12.76e-05 & 8.96e-05 & 3.14e-05 & 3.20e-05 \\ 
      \end{tabular}
        
    \end{table}
    
    The results of the Grace target are shown in more detail in table \ref{tab:grace_segsize}. 
    The top row shows the illumination of the room and the one below shows the absolute difference between two successive results. 
    The last row contains the error $e_d$ that corresponds to the difference images above. It is calculated exactly like $e_r$.
    We can see that a configuration with 24, 48 and 96 lamps produces almost the exact same global result.
    In the case of uniform lighting (1 lamp) or individually controlled sticks (8 lamps) the LEDs facing the ceiling cannot be controlled independent from those facing the walls.
    Because our method searches for a global solution, the ceiling it lit too bright and the walls are missing light.
        
    If we increase the number of lamps we introduce ambiguity: Two successive lamps on a stick produce nearly the same illumination.
    This leads to a separation of the light color and our lamps begin to emit pure light.
    This does not affect the global impression, but it causes colorful highlights.
    The pure colors are reflected by specular objects and stand out between the uniformly lit areas.
    Our optimization does not "see" them because they are very small in size and get overpowered by large uniform areas when searching for a global solution.
    
    This is another reason why we think 24 lamps is the best choice for our lighting system.
    
   \newpage

\section{Sampling}
 \label{eval:sampling}
 
  We tried our nearest-neighbor sampling with different number of sampling directions $|D|$.
  For comparison, we included the limit case where a neighborhood consists of a single pixel
  Table \ref{tab:sampling} show the result error $e_r$ for several targets. The target scaling was $s=1.0$ in all cases.

  \npdecimalsign{.}
    \nprounddigits{4}
    \begin{table}[H]
     \caption[Evaluation of sampling configurations]{\label{tab:sampling} Result error $e_r$ for different sampling configurations, including the limit case containing all available pixels. The lighting system was configured with 24 lamps.}
     \begin{tabular}{c|n{2}{4}|n{2}{4}|n{2}{4}|n{2}{4}|n{2}{4}}
        \toprule 
        {|D|} & {Grace} & {Outdoor} & {Dots} & {CMY} & {Video (avg.)} \\
        \midrule
        \input ../../Results/tables/sampling.table
        \bottomrule
     \end{tabular}
    \end{table}
    \npnoround
    
    We can see the regularization effect of our sampling: Because we average multiple directions, shadows and highlights are removed. 
    The downsampled environment map does not contain as much extreme values as the original. The variance is reduced and the optimizer can find a better solution.
    This causes the result error to decrease with an increased neighborhood size.
    
    We decided that a sampling with 128 direction is perfectly adequate for our system. 

  
\section{Run-time}
 \label{eval:runtime}

  The run-time of our method depends on the number of lamps $|L|$ and the number of sampling directions $|D|$ used. 
  The following table shows the duration of the optimization process in milliseconds.

       
    \npdecimalsign{.}
    \nprounddigits{2}
    \begin{table}[H]
     \caption[Optimization run-times]{\label{tab:optrt}Optimization run-time for the Outdoor target in milliseconds. The other targets showed the same performance.}
     \begin{tabular}{c|n{4}{2}n{4}{2}n{4}{2}n{4}{2}n{4}{2}|n{4}{2}}
        \toprule 
        {|L|} & \multicolumn{1}{r}{33} & \multicolumn{1}{r}{63} & \multicolumn{1}{r}{128} & \multicolumn{1}{r}{291} & \multicolumn{1}{r}{449} & \multicolumn{1}{r}{All}  \\
        \midrule
        \input ../../Results/tables/canon3_lamp-dir_rt_opt.table
        \bottomrule
     \end{tabular}
    \end{table}
    \npnoround
    
    The run-time is largely influenced by the number of lamps $|L|$ and the time complexity of our method lies in $O(|L|^2)$. 
    Increasing the number sampling directions $|D|$ does not increase the run-time because the QP solver benefits from an overdetermined system.
    
    We can see that the configuration with 24 lamps is fast enough for real-time ambient light transfer.
    

\section{Scaling}
 \label{eval:scaling}
 
 The target scaling factor $s$ influences the brightness of the results. It defines the relation between the exposure of the Capture Probe and the exposure of the Calibration Probe.
 The target environment map must be scaled in such a way, that its values are reachable by the optimization. 
 If they are too large, our lamps cannot reproduce the desired brightness. As soon as the lamps reach their maximum intensity, the color can change.
 
 The following tables \ref{tab:scaling_grace} and \ref{tab:scaling_dots} show the influence of the scaling factor on the result.
 
 
   \begin{table}[H]
     \caption[Evaluation of the scaling factor (1)]{\label{tab:scaling_grace}Optimization error $e_o$, results and lamp values for different scale factors $s$ using the Grace target.} 
     \begin{tabular}{cccccc}
        \multicolumn{6}{c}{\includegraphics[width=0.75\linewidth]{../../resources/plots/grace_scaling_plot.svg}} \\
         0.5 & 0.75 & 1.0 & 1.5 & 2.0 & 3.0 \\
         \includegraphics[width=0.1\linewidth]{../../resources/scaling/grace_scaling_0.5_result.png}
		 & \includegraphics[width=0.1\linewidth]{../../resources/scaling/grace_scaling_0.75_result.png}
		 & \includegraphics[width=0.1\linewidth]{../../resources/scaling/grace_scaling_1.0_result.png}
		 & \includegraphics[width=0.1\linewidth]{../../resources/scaling/grace_scaling_1.5_result.png}
		 & \includegraphics[width=0.1\linewidth]{../../resources/scaling/grace_scaling_2.0_result.png}
		 & \includegraphics[width=0.1\linewidth]{../../resources/scaling/grace_scaling_3.0_result.png} \\
		 \includegraphics[width=0.1\linewidth]{../../resources/scaling/grace_scaling_0.5_lamps.png}
		 & \includegraphics[width=0.1\linewidth]{../../resources/scaling/grace_scaling_0.75_lamps.png}
		 & \includegraphics[width=0.1\linewidth]{../../resources/scaling/grace_scaling_1.0_lamps.png}
		 & \includegraphics[width=0.1\linewidth]{../../resources/scaling/grace_scaling_1.5_lamps.png}
		 & \includegraphics[width=0.1\linewidth]{../../resources/scaling/grace_scaling_2.0_lamps.png}
		 & \includegraphics[width=0.1\linewidth]{../../resources/scaling/grace_scaling_3.0_lamps.png} \\

     \end{tabular}
    \end{table}
    
   \begin{table}[H]
     \caption[Evaluation of the scaling factor (2)]{\label{tab:scaling_dots}Optimization error $e_o$, results and lamp values for different scale factors $s$ using the Dots target.}
     \begin{tabular}{cccccc}
        \multicolumn{6}{c}{\includegraphics[width=0.75\linewidth]{../../resources/plots/dots_scaling_plot.svg}} \\
         0.5 & 0.75 & 1.0 & 1.5 & 2.0 & 3.0 \\
         \includegraphics[width=0.1\linewidth]{../../resources/scaling/dots_scaling_0.5_result.png}
		 & \includegraphics[width=0.1\linewidth]{../../resources/scaling/dots_scaling_0.75_result.png}
		 & \includegraphics[width=0.1\linewidth]{../../resources/scaling/dots_scaling_1.0_result.png}
		 & \includegraphics[width=0.1\linewidth]{../../resources/scaling/dots_scaling_1.5_result.png}
		 & \includegraphics[width=0.1\linewidth]{../../resources/scaling/dots_scaling_2.0_result.png}
		 & \includegraphics[width=0.1\linewidth]{../../resources/scaling/dots_scaling_3.0_result.png} \\
		 \includegraphics[width=0.1\linewidth]{../../resources/scaling/dots_scaling_0.5_lamps.png}
		 & \includegraphics[width=0.1\linewidth]{../../resources/scaling/dots_scaling_0.75_lamps.png}
		 & \includegraphics[width=0.1\linewidth]{../../resources/scaling/dots_scaling_1.0_lamps.png}
		 & \includegraphics[width=0.1\linewidth]{../../resources/scaling/dots_scaling_1.5_lamps.png}
		 & \includegraphics[width=0.1\linewidth]{../../resources/scaling/dots_scaling_2.0_lamps.png}
		 & \includegraphics[width=0.1\linewidth]{../../resources/scaling/dots_scaling_3.0_lamps.png} \\ 
         
     \end{tabular}
    \end{table}


  Both plots show how the optimization error $e_o$ varies with the scaling factor $s$. The curve can be divided in three segments with different characteristics:
  For small values $s$, the error is constant. A minimal number of lamps are turned on and their brightness increases in a linear fashion. 
  If we increase $s$, the optimizer turns on more lamps and the error rises continuously. 
  For large values we can observe a shift in the color of the result. Some color channels have reached their maximum intensity and our system is no longer able to reproduce ambient light with the desired brightness..
 

