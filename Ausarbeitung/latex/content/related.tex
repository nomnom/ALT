

\chapter{Related Work}

 
 Digitally controlled full-color lighting can be used to extend the FOV of displays by illuminating the space around the screen with light that matches the displayed scene.
% It aids the perception and increases the user immersion, for example in a movie theater or a living room.
 We can also apply this idea to a whole room: A spatially illuminated room can supply meaningful information to the peripheral vision of the viewer and 
 can increase the feeling of immersion \cite{ghosh2005real}.
 
 The method we present in this paper is based on the Inverse Relighting problem. 
 Our solution is thus closely related to Implicit Relighting and Lighting Design.
    
\section{Relighting and Lighting Design}
 
 There are several Implicit Relighting methods \cite{matusik2004progressively} \cite{fuchs2005bayesian} that use probe objects in order to do relighting with low-frequency illuminations.
 They could be directly applied to our problem by \emph{relighting the room} using the lamps.
 The rendering-part can be omitted as we are only interested in the image weights, which correspond to the brightness of the lamps.
 However, those techniques are designed to render \emph{good} images and are not suited for real-time applications.    
 They also do not use constrained weights which are required in our application, because real lamps have a maximum brightness.
 
 The Inverse Relighting problem is also addressed in Image-Based Lighting Design \cite{anrys2004image} by Anrys et al.:
 They use a light stage to illuminate an object with RGB point-sources and capture the effect of each lamp on the object with a camera from a fixed view-point. 
 The user can then specify the desired illumination of the object by drawing on it in image-space. 
 This serves as a target for the minimization algorithm which finds the weights for the lamps that reproduce the desired illumination. 
 They employ Sequential Quadratic Programming (SQP) to minimize the least squares equation (\ref{eq:minproblem}).
 Analytical representations of the Hessian and Jacobin matrices are employed for an additional speedup.
  
 Our method is closely related to the work of Anrys et al: 
 Instead of illuminating an object, we illuminate the walls of a room and capture omnidirectional images with a light probe.
 This is basically the concept of a light stage turned inside out.
 Instead of SQP, we apply Quadratic Programming (QP) to solve the same constrained least squares problem,
 and reduce the problem size beforehand by performing a downsampling on the light probe data. 

\section{Extending Displays}
    
 Movie and home theaters benefit from dim ambient light and large screens because this setup increases the immersion of the viewer. 
 The screen produces a large image on the viewers retina and the absence of ambient light diminishes false impressions in the peripheral vision.
 Rather than shutting down the ambient light, one can also actively control it to match the brightness and color of the displayed images. 
 
 
 \begin{figure}[H]
  \subfloat[Ambilight\label{fig:ambilight}]{\includegraphics[height=0.25\linewidth]{../../resources/ambilight.png}}
  \hfill
  \subfloat[Illumiroom\label{fig:illumiroom}]{ \includegraphics[height=0.25\linewidth]{../../resources/illumiroom.png} }
  \caption[Ambilight and Illumiroom]{Extending the apparent size of a display by augmenting the surrounding space with light. Philips Ambilight (a) uses LEDs and Microsoft's Illumiroom uses a projector (b).}
 \end{figure}
  
 
 Philips created their Ambilight TVs around this idea: Lines of RGB-LEDs are placed on the edges of a screen and illuminate the wall behind it. 
 The LEDs are controlled to match the average color in the border regions of the image and thus extend the screen onto the wall.
 This gives the impression of a larger television screen and also supplies the room with ambient light of an appropriate hue.
 A study showed that their implementation is well received by the user and greatly improves the feeling of immersion \cite{ambistudy}.
 The Illumiroom concept \cite{jones2013illumiroom} follows a similar idea: A projector is used instead of LEDs to augment the the area around a screen with content of lower resolution.
 
 Just recently Philips introduced wireless controlled LED-based full-color lamps to the consumer market \cite{HUE}. 
 The Philips hue bulbs are shaped like traditional lamps and fit in standard sockets. 
 They can be individually controlled via a wireless station and a smartphone which gives the user full control over hue and brightness.
 More interestingly, they can be linked to the Philips Ambilight TVs so all of the ambient illumination in a room is under control of the television.
 The user can define the approximate location of the lamps and the display in the room.
 The Ambilight TV uses this information to extract appropriate lamp colors from the video stream.
 How exactly this is done has not been published.
 

\section{Real Illumination from Virtual Environments}
  
 Gosh et al. addresses the same problem we do in their work \emph{Real Illumination from Virtual Environments} \cite{ghosh2005real}.
 They too illuminate a room with multiple RGB lamps to reproduce an illumination given by an environment map. 
 However, their work is focused on the effects of ambient light on the user. 
 Their solution is not general and requires an optimal room and lighting setup.
 
 The proposed method works like this: First, they switch the lamps to white one at a time and capture the light spots on the walls with a light probe. 
 They fit a Gaussian to each spot: The center is placed at the the brightest point and the standard deviation is adjusted until the error is minimal.
 They then calculate the color for the lamps by sampling the environment map with the fitted Gaussian kernel.
 This technique only works with calibrated RGB Lamps with a known white point.
 Gosh et al. link the color channels of the LEDs to the color channels of the display, which prohibits the use of white LEDs or other broad-spectrum illuminants.
 There are also cases where a Gauss kernel cannot be fitted with a small error, for example if a lamp causes multiple light spots.
 
 %Our method aims for a more general solution that considers the \emph{global effect} of the lamps and takes all reflected light into account.
